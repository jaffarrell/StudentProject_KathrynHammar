\documentclass[12pt]{article}

\usepackage{amsmath}
\usepackage[left=1.2in,right=1in,top=.5in,bottom=.8in]{geometry}
\usepackage{cancel}
\usepackage{mathtools}

\begin{document}
	\title{Frame Transformation: Vehicle to Earth}
	\author{K. E. Hammar}
	\maketitle
	Some of the terminology used in this document may be used incorrectly as the author is not 100\text{\%} certain of what she is doing, quite yet. 
	\section{Introduction}
		The following information is meant to illustrate the process of computing the reference frame transformation, from vehicle (V) frame to earth (E) frame, of a two-wheel ground vehicle moving on the ground plane (D = 0). This is done through a series of equations to be enumerated in Section 3.
	\section{Variables}
		\begin{table}[hbt]
			\begin{center}
				\begin{tabular}{|c|c|c|}
					\hline
					Symbol & Units & Description \\ \hline
					$e_L$ & $pulse$ & pulses of left wheel encoder\\
					$e_R$ & $pulse$ & pulses of right wheel encoder\\
					$T$ & $s$ & time step size\\
					$E_L$ & $\frac{pulse}{s}$ & pulses of left wheel encoder over the time step \\
					$E_R$ & $\frac{pulse}{s}$ & pulses of right wheel encoder over the time step \\
					$\omega_L$ & $\frac{rad}{s}$ &rate of rotation of left wheel \\
					$\omega_R$ & $\frac{rad}{s}$ & rate of rotation of right wheel \\
					$v_L$ & $\frac{m}{s}$ & velocity of left wheel \\
					$v_R$ & $\frac{m}{s}$ & velocity of right wheel \\
					$u_v$ & $\frac{m}{s}$ & velocity of vehicle \\
					$\omega_v$ & $\frac{rad}{s}$ & rate of rotation of vehicle \\
					$R_L$ & $m$ & radius of left wheel\\
					$R_R$ & $m$ & radius of right wheel\\
					$L$ & $m$ & length of vehicle axle\\
					$\psi_v$ & $rad$ & yaw angle of vehicle\\
					$\dot{\psi}_v$ & $\frac{rad}{s}$ & rate of rotation of vehicle \\ 
					$T_{EV}$ & $m$ & position vector from origin of E-frame to origin of V-frame\\
					$\dot{T}_{EV}$ & $\frac{m}{s}$ & velocity vector of vehicle in E-frame 
				\\ \hline
			\end{tabular}
			\end{center}
		\end{table}
	\section{Equations}
		\begin{eqnarray}
		\omega_L &=& \frac{2\pi}{N} E_L\\
		\omega_R &=& \frac{2\pi}{N} E_R\\
		v_L &=& R_L\cdot\omega_L\\
		v_R &=& R_R\cdot\omega_R\\
		u_v &=& \frac{v_L + v_R}{2}\\
		\omega_v &=& \frac{1}{L}(v_L - v_R)\\
		\dot{\psi}_v &=& \omega_v\\
		\dot{T}_{EV} &=& 
		\left[
		\begin{array}{c}
		\cos \psi_v\\
		\sin \psi_v \\
		0
		\end{array}
		\right] u_v\\
		E_L &=& \frac{e_L}{T}\\
		E_R &=& \frac{e_R}{T}
	\end{eqnarray}
	Using the equations above as a foundation, we can progress to the first step of the transformation process. We need to find $\psi_k$ and ${T}_{EV}$, we will do this using Euler's Method.
	\begin{eqnarray}
	\psi_v(k) &=& \psi_v(k-1) + \omega_v\cdot T \nonumber\\
	&=& \psi_v(k-1) + \frac{1}{L}(v_L - v_R)\cdot T \nonumber\\
	&=& \psi_v(k-1) + \frac{1}{L}(R_L\cdot\omega_L - R_R\cdot\omega_R)\cdot T \nonumber\\
	&=& \psi_v(k-1) + \frac{1}{L}\left(R_L\left(\frac{2\pi}{N} E_L\right) - R_R\left(\frac{2\pi}{N} E_R\right)\right)\cdot T \nonumber\\
	&=& \psi_v(k-1) + \frac{1}{L}\left(R_L\left(\frac{2\pi}{N} \cdot \frac{e_L}{\cancel{T}}\right) - R_R\left(\frac{2\pi}{N} \cdot  \frac{e_R}{\cancel{T}}\right)\right)\cdot \cancel{T} \nonumber\\
	&=& \psi_v(k-1) + \frac{1}{L}\left(R_L\left(\frac{2\pi \cdot e_L}{N}\right) - R_R \left(\frac{2\pi \cdot e_R}{N}\right)\right) \nonumber \\
	&=& \psi_v(k-1) + \frac{2\pi}{N \cdot L}(R_L \cdot e_L - R_R \cdot e_R)\\
	\nonumber \end{eqnarray}
	We have now found $\psi_v$ in terms of only estimable and known values: a previous yaw angle ($\psi_v(k-1)$), the pulse value of both encoders ($e_{L,R}$), the radius of both wheels ($R_{L,R}$) and the length of the axle ($L$). Using (1)-(5) and (8)-(10) we will complete the same process for ${T}_{EV}(k)$, the position vector.\\
	\begin{eqnarray}
	{T}_{EV}(k) &=& {T}_{EV}(k-1) + \left[
	\begin{array}{c}
	\cos \psi_v\\
	\sin \psi_v \\
	0
	\end{array}
	\right] u_v \cdot T \nonumber \\
	&=& {T}_{EV}(k-1) + \left[
	\begin{array}{c}
	\cos \psi_v\\
	\sin \psi_v \\
	0
	\end{array}
	\right] \left( \frac{v_L + v_R}{2} \right) \cdot T \nonumber \\
	&=& {T}_{EV}(k-1) + \left[
	\begin{array}{c}
	\cos \psi_v\\
	\sin \psi_v \\
	0
	\end{array}
	\right] \frac{1}{2}(R_L\cdot\omega_L + R_R\cdot\omega_R)\cdot T \nonumber \nonumber \\
	&& \text{As seen in the derivation of $\psi_v$ this can be simplified to:}	\nonumber \\
	&=& {T}_{EV}(k-1) + \left[
	\begin{array}{c}
	\cos \psi_v\\
	\sin \psi_v \\
	0
	\end{array}
	\right] \frac{1}{2}\left(R_L\left(\frac{2\pi \cdot e_L}{N}\right) + R_R\left(\frac{2\pi \cdot e_R}{N}\right)\right) \nonumber  \\
	&=& {T}_{EV}(k-1) + \left[
	\begin{array}{c}
	\cos \psi_v\\
	\sin \psi_v \\
	0
	\end{array}
	\right] \frac{\pi}{N}(R_L\cdot e_L + R_R\cdot e_R) \\
	\nonumber \end{eqnarray}
	
	\section{State Space}
	
	\begin{eqnarray}
	\text{State Vector:}&& \nonumber\\
	x &=& \left[
	\begin{array}{c}
	T_{EV}\\
	\psi_v 
	\end{array}
	\right]\\
	\text{Input:}&& \nonumber \\
	u &=& \left[
	\begin{array}{c}
	e_L\\
	e_R 
	\end{array}
	\right]\\
	\text{State Space:}&& \nonumber\\
	\dot{x} &=& Ax+Bu\\
	y &=& Cx\\
	\nonumber \\
	\dot{x} &=& \left[
	\begin{array}{c}
	\dot{T}_{EV}\\
	\dot{\psi}_v 
	\end{array}
	\right] \nonumber \\
	&=& \left[
	\begin{array}{c}
	\cos(\psi_v)\cdot u_v\\
	\sin(\psi_v)\cdot u_v\\
	\omega_v 
	\end{array}
	\right] \nonumber\\
	\nonumber \\
	\nonumber \\
	let, \left[
	\begin{array}{c}
	\cos(\psi_v)\\
	\sin(\psi_v) 
	\end{array}
	\right] &=& \left[
	\begin{array}{c}
	1\\
	0
	\end{array}
	\right] \nonumber \\
	\nonumber \\
	\nonumber \\
	\dot{x} &=& \left[
	\begin{array}{c}
	1\cdot u_v\\
	0\\
	\omega_v 
	\end{array}
	\right] \nonumber \\
	&=& \left[
	\begin{array}{c}
	\frac{\pi}{N\cdot T}(R_L\cdot e_L + R_R\cdot e_R)\\
	0\\
	\frac{2\pi}{N \cdot L\cdot T}(R_L \cdot e_L - R_R \cdot e_R) 
	\end{array}
	\right] \\
	\nonumber \\
	y &=& \left[
	\begin{array}{c}
	T_{EV} 
	\end{array}
	\right]\\
	\nonumber \\
	\text{Matrix Form:} && \nonumber\\
	A &=& \left[
	\begin{array}{ccc}
	0&0&0\\
	0&0&0\\
	0&0&0\\
	\end{array}
	\right]\\	
	\text{This looks incorrect.}&&\text{Why would A be all zeros?}\nonumber \\
	B &=& \left[
	\begin{array}{cc}
	\frac{\pi\cdot R_L}{N\cdot T}&\frac{\pi\cdot R_R}{N\cdot T}\\
	0&0\\
	\frac{2\pi\cdot R_L}{N \cdot L\cdot T}&\frac{-2\pi\cdot R_R}{N \cdot L\cdot T}\\
	\end{array}
	\right]\\
	C &=& \left[
	\begin{array}{ccc}
	1&1&0\\
	\end{array}
	\right]\\	
	\nonumber \end{eqnarray}
	
	 
\end{document}